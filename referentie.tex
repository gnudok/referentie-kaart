%%=====================================================================================
%%
%%       Filename:  referentie.tex
%%
%%    Description:  
%%
%%        Version:  1.0
%%        Created:  11-06-15
%%       Revision:  none
%%
%%         Author:  YOUR NAME (), 
%%   Organization:  
%%      Copyright:  
%%
%%          Notes:  
%%                
%%=====================================================================================

\documentclass[10pt]{article}
\usepackage{fixltx2e}
\usepackage[orthodox,l2tabu,abort]{nag}
\usepackage{amsmath}

% Page layout
\usepackage[landscape,margin=0.5in]{geometry}
\usepackage{multicol}

% Title area
\usepackage{titling} % Allows for use of date, author, etc. after \maketitle
% Ref: http://tex.stackexchange.com/questions/3988/titlesec-versus-titling-mangling-thetitle
\let\oldtitle\title
\renewcommand{\title}[1]{\oldtitle{#1}\newcommand{\mythetitle}{#1}}
\renewcommand{\maketitle}{%
{\begin{center}\Large \mythetitle\end{center}}
}

% Document divisions
\usepackage{titlesec}
\setcounter{secnumdepth}{0}
\titlespacing{\section}{0pt}{0pt}{0pt}
\titlespacing{\subsection}{0pt}{0pt}{0pt}
\usepackage{nopageno} % To keep \section from resetting page style

\setlength{\parindent}{0pt} % disabling indentation by default

% Lists
\usepackage{enumitem} % for consistent formatting of lists
\newlist{ttdesc}{description}{1}
\setlist[ttdesc]{font=\ttfamily,noitemsep}
\usepackage{calc} % for \widthof

% Code
\usepackage{listings}
\lstset{language=[LaTeX]TeX,%
	basicstyle=\itshape,%
	keywordstyle=\normalfont\ttfamily,%
	morekeywords={part,chapter,subsection,subsubsection,paragraph,subparagraph}%
}

\usepackage{lipsum}

\title{GNUDok referentie kaart}
\author{Steven L. Speek}
\date{\today{}}

\begin{document}
\begin{multicols}{3}
	\maketitle

	\section{LXDE Desktop}
	\begin{ttdesc}[labelwidth=\widthof{\texttt{ALT-SHIFT-$\leftarrow$}}]
	\item[ALT-CTRL-$\leftarrow$] Desktop naar links
	\item[ALT-CTRL-$\rightarrow$] Desktop naar rechts
	\item[ALT-CTRL-$\downarrow$] Desktop naar beneden
	\item[ALT-CTRL-$\uparrow$] Desktop naar boven
	\item[SHIFT-ALT-$\leftarrow$] Verplaats naar desktop links
	\item[SHIFT-ALT-$\rightarrow$]Verplaats naar desktop rechts
	\item[SHIFT-ALT-$\downarrow$] Verplaats naar desktop beneden
	\item[SHIFT-ALT-$\uparrow$] Verplaats naar desktop boven
	\item[WIN-F1] Ga naar desktop 1
	\item[WIN-F2] Ga naar desktop 2
	\item[WIN-F3] Ga naar desktop 3
	\item[WIN-F4] Ga naar desktop 4
	\item[WIN-D of ALT-CTRL-D] Minimaliseer alle vensters
	\item[ALT-F4] Sluit venster
	\item[ALT-F10] Maximaliseer venster
	\item[CTRL-WIN-Q] Zet computer direct uit
	\end{ttdesc}
	\section{Toepassingen openen}
	\begin{ttdesc}[labelwidth=\widthof{\texttt{WIN-E or ALT-F2}}]
	\item[WIN-C] Chromium browser
	\item[WIN-E] Bestandsbeheer
	\item[WIN-F] Zoeken
	\item[WIN-I] Iceweasel browser
	\item[WIN-K] Xkill
	\item[WIN-L] Libreoffice
	\item[WIN-P] Popcorn Time
	\item[WIN-R or ALT-F2] Run box
	\item[WIN-T] Terminal
	\item[CTRL-D] Verlaat terminal
	\item[WIN-V] VLC-speler
	\item[WIN-X] Kodi
	\item[WIN-Z] Calculator
	\end{ttdesc}
	\section{Generiek}
	\begin{ttdesc}[labelwidth=\widthof{\texttt{SHIFT+Tab}}]
	\item[Tab] Focus volgend schermcomponent
	\item[SHIFT-Tab] Focus vorig schermcomponent
	\item[Space] Druk knop in
	\item[ALT-$\downarrow$] Open dropdown lijst
	\item[CTRL-C] Kopieer naar clipbord
	\item[CTRL-X] Knip naar clipbord
	\item[CTRL-V] Plakken
	\item[CTRL-P] Printen
	\end{ttdesc}
	\section{Chromium}
	\begin{ttdesc}[labelwidth=\widthof{\texttt{CTRL+SHIFT-Tab}}]
	\item[CTRL-L] Focus locatiebalk
	\item[CTRL-T] Open nieuw tabblad
	\item[CTRL-W of CTRL-F4] Sluit huidig tabblad
	\item[CTRL-Tab of CTRL-PAGE DOWN] Volgend tabblad
	\item[CTRL-SHIFT-Tab of CTRL-PAGE UP] Vorig tabblad
	\item[F5] Herlaad pagina
	\item[CTRL-D] Bladwijzer maken
	\item[CTRL-F] Zoeken in de pagina
	\item[CTRL-SHIFT-B] Bladwijzerbalk zichtbaarheid
	\item[F10 of ALT-F] Focus menu knop
	\item[F6] Focus naar volgend gebied
	\item[SHIFT-F6] Focus naar vorig gebied
	\end{ttdesc}
	\section{APOD-BG}
	\begin{ttdesc}[labelwidth=\widthof{\texttt{WIN-SHIFT+E}}]
	\item[ALT-CTRL-M] Wissel view-mode
	\item[ALT-CTRL-J] Volgende achtergrond
	\item[ALT-CTRL-K] Vorige achtergrond
	\item[ALT-CTRL-R] Random eerdere achtergrond
	\item[ALT-CTRL-I] Open APOD-pagina voor achtergrond
	\item[ALT-CTRL-A] Open APOD-pagina
	\end{ttdesc}

	\section{BASH}
	\begin{ttdesc}[labelwidth=\widthof{\texttt{Tab Tab}}]
	\item[Tab] Probeer huidig woord aan te vullen
	\item[Tab Tab] Toon woorden die huidig woord kunnen aanvullen
	\item[CTRL+P of $\uparrow$] Blader vorige commando's
	\item[CTRL+N of $\downarrow$] Blader volgende commando's
	\item[ALT+.] Blader vorige laatste argumenten
	\item[CTRL+D] Verwijder volgende character, of sluit huidige sessie af
	\item[ALT+D] Verwijder volgend woord
	\item[CTRL+R] Zoek terug in historie
	\item[SHIFT+PAGE UP] Scroll pagina omhoog
	\item[SHIFT+PAGE DOWN] Scroll pagina omlaag
	\item[CTRL+B] Ga \'{e}\'{e}n teken terug
	\item[CTRL+F] Ga \'{e}\'{e}n teken vooruit
	\item[ALT+B] Ga \'{e}\'{e}n woord terug
	\item[ALT+F] Ga \'{e}\'{e}n woord vooruit
	\item[CTRL+A] Ga naar het einde van de regel
	\item[CTRL+E] Ga naar het begin van de regel
	\item[CTRL+K] Knip achter de cursor 
	\item[CTRL+U] Knip voor de cursor 
	\item[CTRL+W] Knip vorig woord 
	\item[CTRL+Y] Plakken 
	\end{ttdesc}
	%\section{Filler material}

	%\lipsum

	\noindent Copyright \textcopyright{} \thedate{} \theauthor{}

\end{multicols}
\end{document}

